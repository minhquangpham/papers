% This must be in the first 5 lines to tell arXiv to use pdfLaTeX, which is strongly recommended.
\pdfoutput=1
% In particular, the hyperref package requires pdfLaTeX in order to break URLs across lines.

\documentclass[11pt]{article}

% Remove the "review" option to generate the final version.
\usepackage[review]{emnlp2021}

% Standard package includes
\usepackage{times}
\usepackage{latexsym}

% For proper rendering and hyphenation of words containing Latin characters (including in bib files)
\usepackage[T1]{fontenc}
% For Vietnamese characters
% \usepackage[T5]{fontenc}
% See https://www.latex-project.org/help/documentation/encguide.pdf for other character sets

% This assumes your files are encoded as UTF8
\usepackage[utf8]{inputenc}
\renewcommand{\UrlFont}{\ttfamily\small}
\usepackage{graphicx}
\usepackage{grffile}
\usepackage{multirow}
\usepackage{xcolor,colortbl}
\usepackage{amsmath}
%\usepackage{fdsymbol}
\usepackage{amssymb}
\usepackage{makecell}
\usepackage[super]{nth}
\usepackage{arydshln}
\usepackage{algorithm,algorithmic}
%\usepackage[noend]{algpseudocode}
\usepackage{regexpatch}
\usepackage{subcaption}
% This is not strictly necessary, and may be commented out,
% but it will improve the layout of the manuscript,
% and will typically save some space.
\usepackage{microtype}

\usepackage{url}
\usepackage{longtable}
\usepackage{tikz}
\usetikzlibrary{calc}
\usepackage[draft]{todo}
\usepackage[normalem]{ulem}
\usepackage{xspace}
\usepackage{float}

\usepackage{pgfplots}
\usepackage{pgfplotstable}

\let\oldbibitem\bibitem
\def\bibitem{\vfill\oldbibitem}

\usepackage{soulutf8}
\usepackage{tabularx}

\usepackage{multirow}
\usepackage{adjustbox}
\usepackage{caption}
\usepackage{subcaption}

\newcommand{\fyTodo}[1]{\Todo[FY:]{\textcolor{orange}{#1}}}
\newcommand{\fyTodostar}[1]{\Todo*[FY:]{\textcolor{orange}{#1}}}
\newcommand{\fyDone}[1]{\done[FY]\Todo[FY:]{\textcolor{orange}{#1}}}
\newcommand{\fyFuture}[1]{\done[FY]\Todo[FY:]{\textcolor{red}{#1}}}
\newcommand{\fyDonestar}[1]{\done[FY]\Todo[FY:]{\textcolor{orange}{#1}}}

\newcommand{\revision}[1]{\textcolor{red}{#1}}
\newcommand{\revisiondel}[1]{}
\newcommand{\src}{\ensuremath{\mathbf{f}}} % source sentence
\newcommand{\trg}{\ensuremath{\mathbf{e}}} % target sentence
\newcommand{\domain}[1]{\texttt{\textsc{#1}}}
\newcommand{\system}[1]{\texttt{{#1}}}

\newcommand{\vlambda}{\ensuremath{\boldsymbol\lambda}\xspace} % parameters vector for a distribution
\newcommand{\vtheta}{\ensuremath{\boldsymbol\theta}\xspace} % parameters vector for a distribution
\newcommand{\vpsi}{\ensuremath{\boldsymbol\psi}\xspace} % parameters vector for a distribution
\newcommand{\indic}[1]{\ensuremath{\mathbb{I}(#1)}}
% \newcommand{\SB}[1]{\textcolor{green}{#1}}
% \newcommand{\SW}[1]{\textcolor{red}{#1}}
\newcommand{\SB}[1]{\textbf{#1}}
\newcommand{\SW}[1]{\underline{#1}}
% limits underneath
\DeclareMathOperator*{\argmin}{arg\,min}
\DeclareMathOperator*{\argmax}{arg\,max}
\renewcommand\textfraction{.1}
\renewcommand\floatpagefraction{.95}
\newcommand{\sbcl}[2]{{\scriptsize #1 \hfill $|$ \hfill  #2}}
\usepackage{multirow}
\usepackage{adjustbox}

\title{Supplementary materials}

\author{First Author \\
  Affiliation / Address line 1 \\
  Affiliation / Address line 2 \\
  Affiliation / Address line 3 \\
  \texttt{email@domain} \\\And
  Second Author \\
  Affiliation / Address line 1 \\
  Affiliation / Address line 2 \\
  Affiliation / Address line 3 \\
  \texttt{email@domain} \\}

\date{}

\begin{document}
\maketitle
\setlength{\abovedisplayskip}{2pt}
\setlength{\belowdisplayskip}{2pt}

\section*{A - Generalized Multi-Domain Dynamic Adaptation Curriculum Algorithm}
We provide below in the pseudo algorithm \ref{alg:mdmt}  the generalized Multi-Domain Adaptation Dynamic Sampling Algorithm.
\begin{algorithm}[htbp]
\caption{Multi-Domain Adaptation Dynamic Sampling} \label{alg:mdmt}
\begin{algorithmic}[1]
\REQUIRE {
\begin{itemize}
	\item K corpora $C^d, d\in [1,..,K]$ for $K$ domains equipped by an empirical distribution $D_d(x)$
	\item K dev sets $Dev^d, d\in [1,..,K]$ for $K$ domains.
	\item Domain testing distribution $\lambda^t \in \mathbb{R}^K$
	\item Batch size $B$
	\item Domain Dynamic Sampling Distribution $\lambda^l(i) \in \mathbb{R}^K$ where $i$ is the enumeration of the iteration.
	\item $Eval\_scores = []$
	\item $Early\_stopping$ criterion
	\item Total training iterations $iter\_num$
	\item Updating rule of the sampling distribution $\lambda^l(i)$
\end{itemize}}
\REPEAT 
\STATE{Iteration i.}
\STATE{Randomly pick $d \in [1,..,K]$ from sampling distribution $\lambda^l(i)$}
\STATE{Sample $B$ sentences from $C^d$ with empirical distribution $D_d(x)$}
\STATE{Update model by applying SGD computed from $B$ sampled sentences}
\IF{$i \equiv 0 \mod{eval\_step}$}
	\STATE{Evaluate current model with $K$ dev sets. $S^i_d$ is the performance at iteration $i-th$ on domain $d$}
	\STATE{Report weighted score using the domain testing distribution $\lambda^t$. $$eval(i) = \displaystyle{\mathop{\sum}_d^K \lambda^t(d) S^i_d}$$}
	\STATE{$Eval\_scores.append(eval(i))$.}
\ENDIF
\IF{$i \equiv 0 \mod{sampler\_updating\_step}$}
	\STATE Update $\lambda^l(i)$.
\ENDIF
\IF{$Early\_stopping(Eval\_scores)$}
	\STATE{Break.}
\ENDIF
\UNTIL{$i> iter\_num$}
\end{algorithmic}
\end{algorithm}

\section*{B - Experiments with automatic domains}
\begin{table*}[htbp]
  \centering
  \footnotesize
  \begin{tabular}{|p{1.3cm}|*{6}{c|}} \hline
Cluster&MED&ECB&IT&LAW&REL&TALK \\
\hline
1&0.24&0.11&0.47&0.17&0.00&0.01 \\
2&0.48&0.04&0.19&0.28&0.00&0.00 \\
3&1.00&0.00&0.00&0.00&0.00&0.00 \\
4&0.05&0.06&0.01&0.87&0.00&0.00 \\
5&0.03&0.00&0.00&0.00&0.86&0.10 \\
6&0.44&0.04&0.40&0.07&0.01&0.04 \\
7&0.92&0.01&0.01&0.05&0.00&0.00 \\
8&0.98&0.00&0.00&0.02&0.00&0.00 \\
9&0.02&0.01&0.96&0.01&0.00&0.00 \\
10&0.98&0.00&0.00&0.02&0.00&0.00 \\
11&0.93&0.00&0.01&0.04&0.00&0.01 \\
12&0.98&0.00&0.01&0.01&0.00&0.00 \\
13&0.99&0.00&0.00&0.01&0.00&0.00 \\
14&0.03&0.37&0.01&0.59&0.00&0.00 \\
15&0.98&0.00&0.01&0.01&0.00&0.00 \\
16&0.02&0.89&0.01&0.08&0.00&0.00 \\
17&0.96&0.01&0.02&0.01&0.00&0.00 \\
18&0.93&0.00&0.05&0.02&0.00&0.00 \\
19&0.99&0.00&0.00&0.00&0.00&0.00 \\
20&0.15&0.00&0.79&0.05&0.00&0.01 \\
21&0.99&0.00&0.00&0.01&0.00&0.00 \\
22&0.21&0.01&0.40&0.04&0.02&0.32 \\
23&0.11&0.48&0.02&0.39&0.00&0.00 \\
24&0.99&0.00&0.01&0.00&0.00&0.00 \\
25&0.95&0.00&0.04&0.01&0.00&0.00 \\
26&0.59&0.02&0.23&0.11&0.00&0.05 \\
27&0.53&0.00&0.02&0.00&0.01&0.44 \\
28&0.99&0.00&0.00&0.01&0.00&0.00 \\
29&0.56&0.16&0.08&0.16&0.00&0.04 \\
30&0.01&0.06&0.00&0.92&0.00&0.00 \\
\hline
\end{tabular}
\caption{Domains's proportion in each cluster.}
\label{tab:automatic_domains}
\end{table*}

This experiment aims to simulate with automatic domains a scenario where the number of ``domains'' is large and where some ``domains'' are close and can effectively share information. Full results in Table~\ref{tab:automatic_domains}. Cluster size vary from approximately 8k sentences (cluster~24) up to more than 350k sentences. More than 2/3 of these clusters mostly comprise texts from one single domain, as for cluster 12 which is predominantly \domain{med}, the remaining clusters typically mix 2 domains.

\end{document}











